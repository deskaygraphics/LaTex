\documentclass[a4paper,10pt]{article}
\usepackage[left=0.3in, right=.3in, top=.3in, bottom=.3in]{geometry}
\usepackage{multicol}
\usepackage{enumitem, hyperref, fontawesome5, multicol}
\usepackage{titlesec}

\titleformat{\section}{\large\bfseries}{}{0em}{}[\titlerule]
\titlespacing*{\section}{0pt}{0.5em}{0.5em}

\begin{document}

\sloppy % Fix overfull/underfull warnings
% Header
\begin{center}
    {\huge \textbf{Desmond Kangah}}\\
    \textit{Msc Civil Engineering}\\
    \textbf{Phone}  +1 2254336635 \quad
    \textbf{Email} \href{desmondkangah98@gmail.com}{desmondkangah98@gmail.com} \quad
    \textbf{GitHub} \href{https://github.com/kangahdesmond}{kangahdesmond}
    \textbf{Linkedin} \href{https://www.linkedin.com/in/desmond-kangah-629b9a27b/}{desmond-kangah}
\end{center}

% Professional Summary
\section*{PROFESSIONAL SUMMARY}
Geospatial engineer and surveyor specializing in GeoAI, using AI and machine learning 
to analyze spatial data, automate processes, and deliver innovative solutions for 
urban planning, environmental monitoring, and infrastructure development. 
Skilled in GIS, remote sensing, and spatial programming.

% Education
\section*{EDUCATION}
\begin{itemize}[leftmargin=*, noitemsep]
    \item \textbf{MSc in Civil Engineering} \hfill Aug. 2024 {-} Present \\
    Louisiana State University, Baton Rouge, LA {--} Concentration: Geodesy and Geomatics \qquad
    \textit{Current GPA 4.0/4.0\@}
    \item \textbf{BSc in Geomatic Engineering}  \quad
    University of Mines and Technology, Tarkwa, Ghana, CWA 81.24/100 \hfill Sep 2018 {-} 2022
\end{itemize}

 \section*{Relevant Courses}
 \begin{multicols}{4} % Three columns
 \begin{itemize}[leftmargin=*]
     \item GIS and Remote Sensing
     \item Geodesy
     \item Photogrammetry
     \item Deep Learning
     \item Machine Learning
     \item Computer Vision
     \item Spatial Software Design
     \item GNSS
 \end{itemize}
 \end{multicols}

 % Skills
\section*{SKILLS}
\textbf{Softwares}
\textbf{Geospatial Tools} Geemap, Leafmap, Google Earth Engine, QGIS, ArcGIS, 
SNAP, ISCE2, GMT, ENVI, SNAP\\
\textbf{Languages:}  HTML/CSS, JavaScript, Python, Matlab\\
\textbf{Tools:} Git, VS Code, PostgreSQL, ISCE2 Tools, Geospatial packages\\
\textbf{Survey Instrument:} Total Station, Dumpy Level, RTK GPS and Drones\\
\textbf{Other Skills:} AutoCAD, Civil 3D, Photoshop, Microsoft Office Suite, LaTeX, Vscode, Spyder, Lyx.


% Experience
\section*{WORK EXPERIENCE}
\begin{itemize}[leftmargin=*, noitemsep]
    \item \textbf{Teaching and Research Assistant} \hfill Aug 2024 {-} Present \\
    Louisiana State University, Baton Rouge, LA
    \begin{itemize}[noitemsep]
        \item Processing and analyzing InSAR data for land subsidence monitoring
        \item Assisted in teaching surveying courses
        \item Conducted research on AI for land use classification
        \item Developing web-based GIS application for urban planning
    \end{itemize}

    \item \textbf{Chief Surveyor} \hfill Jan. 2024 {-} Aug. 2024 \\
    MAC Partners, Mining, Accra, 
    \begin{itemize}[noitemsep]
        \item Conducted topographic surveys for Processing Plant construction
        \item Created CAD 3D models for Plant site planning and development
        \item Setting out of infrastructure and plant designs for construction
        \item Conducted drone surveys for power line routing, and volume calculation
    \end{itemize}

    \item \textbf{Teaching and Research Assistant} \hfill Jun. 2023 {-} Dec. 2023 \\
    University of Mines and Technology, Tarkwa, Ghana
    \begin{itemize}[noitemsep]
        \item Teaching GIS and remote sensing for free
        \item Teaching surveying and it applications using GNSS devices, total stations, and drones
        \item Guided students in their final year projects and Lecturers in their research
    \end{itemize}

    \item \textbf{Assistant Surveyor} \hfill Oct. 2022 {-} May. 2023 \\
    Ghana Highway Authority, Accra, Ghana
    \begin{itemize}[noitemsep]
        \item Conducted road surveys for road construction and maintenance
        \item Created CAD drawings for road designs
        \item Assisted in setting out of road alignments
        \item Conducted drone surveys for road corridor mapping
    \end{itemize}


    \item \textbf{Assistant Surveyor (Part-Time)} \hfill Oct. 2019 {-} Dec. 2021 \\
    Wilhelm Construction, Tarkwa, Ghana
    \begin{itemize}[noitemsep]
        \item Conducted topographic surveys for road construction
        \item Created CAD drawings for road designs
        \item Assisted in setting out of road alignments
        \item Conducted drone surveys for road corridor mapping. %\newline
    \end{itemize}
\end{itemize}

% Projects
\section*{PROJECTS} 
\begin{itemize}[leftmargin=*]
    \item \textbf{InSAR Analysis for Land Subsidence Monitoring} \hfill Sep. 2024 {-} Jan. 2025 \\
    Used InSAR data to monitor deformation in East Baton Rouge Parish Transportation Network. I use sentinel-1 data 
    and processed it using PSI and SBAS techniques. The results were validated using GNSS data and final velocity maps were created.

    \item \textbf{Land Use and Land Cover Segmentation} \hfill Feb. 2025 \\
    Used Unet deep learning to segment land use and land cover from satellite images into classes. The techniques involved
    labeling the data, data augmentation, training the model, and evaluating the model performance.
    A model deployment was done using using Hugging Face to automate segmentation of new images.


    \item \textbf{Landslide Susceptibility Mapping} \hfill Dec. 2024\\
    Used remote sensing and GIS to map landslide susceptibility zones in East Baton Parish
    using AHP and logistic regression. I exploited whitebox, Gee, Geemap and ArcGIS for 
    create the various layers and the final susceptibility map.

    \item \textbf{Common Grid Software} \hfill Nov. 2023 \\
   I did developed this software through a contract with two mining company to automate the process of creating common grids for
   mine coordinating. The software was developed using using Matlab App designer and geodetic formulas and parameters.

    \item \textbf{3D Coordinate Transformation Software} \hfill Jul. 2021 {-} Aug. 2022 \\
    Bsc. Geomatic Engineering final year project. Developed a software to transform 3D coordinates between
    different coordinate systems. The software was developed using C-Sharp and Visual Studio.

    \item \textbf{Water Quality Analysis} \hfill June. 2021 {-} Jul 2022 \\
    In this project, I analyzed the water quality of Lake Bosomtwe using Google Earth Engine. I used Landsat data to
    monitor the water quality of the river. The results were validated using field data and the final map was created using ArcGIS.

    \item \textbf{RainFall Forcast using ARIMA} \hfill Aug. 2022 {-} Dec. 2022 \\
    This project was a contract from Benso Palm Plantation Factory. I used ARIMA to forecast rainfall for the next 5 years using their rainfall collected data for 30 years. The data was collected from the Ghana Meteorological Agency and
    the model was trained and tested using Python. The results were validated using field data and the final forecast was created.
\end{itemize}

%Leadership
\section*{LEADERSHIP}   
\begin{itemize}[leftmargin=*]
    \item \textbf{Vice President, Tertiary Student Association of Ahanta} \hfill Sep. 2021 {-} Sep. 2022 \\
    Organized seminars, workshops, and field trips for students. Represented the Association at meetings and events.
\end{itemize}

%GRANTS
\section*{GRANTS}   
\begin{itemize}[leftmargin=*]
    \item \textbf{Ghana National Petroleum Corporation (GNPC)} \hfill Sep. 2018 {-} Aug. 2022 \\
    Full scholarship for Bsc. Geomatic Engineering at the University of Mines and Technology.
\end{itemize}

\section*{ONLINE COURSES (Certificates)}
\begin{itemize}[leftmargin=*]
    \item \textbf{Applied AI Lab– Worldquant University} \hfill Jan. 2025 {-} Present\\
    In this course, I am mastering Computer vision, Deep learning, Machine learning and Transfer learning. I have already some competition project such as endengered species classification, 
    image segmentation, object detection, and face recognition using YOLO, CNN, Transfer learning on ResNet and Pytorch.

    \item \textbf{Spatial Software Design – University of Tennessee}  \hfill Jan. 2025 {-} Present\\
    In this course, I learned how to design and develop spatial software using Python, JavaScript, and HTML/CSS. I mastered the use of git and github for version control and collaboration. 
    I created my own website using github.io and automated GIS processes using Python.

    \item \textbf{PostgreSQL for Spatial Query – DataCamp and Open Geospatial Solutions} \hfill Jan. 2025 {-} Feb. 2025\\
    In this course, I learned how to use PostgreSQL for spatial queries. I mastered the use of PostGIS for spatial queries, spatial joins, and spatial analysis.

    \item \textbf{Deep Learning for LULC Classification – Kaggle and 650 AI Lab} \hfill Feb. 2025 \\
    In this course, I learned how to use deep learning for land use and land cover classification and segmentation. 
    I mastered the use of Unet, ResNet, and EfficientNet for LULC classification.

    \item \textbf{InSAR Training Course – University of Alaska } \hfill Oct. 2024 {-} Dec. 2024\\
    In this course, I learned the basics of InSAR, how to process InSAR data, and how to interpret the results. I mastered the use of ISCE2 Tools, SNAP, and GMT for InSAR processing.

    \item \textbf{GIS Programming – University of Tennessee} \hfill Sep. 2024 {-} Jan. 2025\\
    In this course, I learned how to use Python to automate GIS processes, making use of Xarray, Rasterio, pandas, and efficient use of GIS tools. I mastered the use of whitebox, Geemap, Leafmap, and Google Earth Engine for GIS programming.


    \item \textbf{Deep and Machine Learning  – Pantech Solutions} \hfill March. 2024  {-} Dec. 2024\\
    In this course, I learned the basics of deep learning, machine learning, and computer vision. I mastered the use of Python, Tensorflow, Keras, and Pytorch for deep learning and machine learning.
    \item \textbf{Google Earth Engine for Geospatial Analysis – Study Hacks} \hfill Jul. 2024 \\
    In this course, I learned how to use Google Earth Engine for geospatial analysis. I mastered the use of Google Earth Engine for land cover classification, change detection, and time series analysis.
    
\end{itemize}
\end{document}






